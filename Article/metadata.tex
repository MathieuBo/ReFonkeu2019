% DO NOT EDIT - automatically generated from metadata.yaml

\def \codeURL{https://github.com/rescience-c/template}
\def \codeDOI{}
\def \codeSWH{}
\def \dataURL{}
\def \dataDOI{}
\def \editorNAME{}
\def \editorORCID{}
\def \reviewerINAME{}
\def \reviewerIORCID{}
\def \reviewerIINAME{}
\def \reviewerIIORCID{}
\def \dateRECEIVED{01 November 2018}
\def \dateACCEPTED{}
\def \datePUBLISHED{}
\def \articleTITLE{[Re] How mRNA localization and Protein Synthesis Sites Influence Dendritic Protein Distribution and Dynamics}
\def \articleTYPE{Replication}
\def \articleDOMAIN{Computational Neuroscience}
\def \articleBIBLIOGRAPHY{bibliography.bib}
\def \articleYEAR{2021}
\def \reviewURL{}
\def \articleABSTRACT{Three years have passed since ReScience published its first article and since September 2015, things have been going steadily. We're still alive, independent and without a budget. In the meantime, we have published around 20 articles (mostly in computational neuroscience \& computational ecology) and the initial has grown from around 10 to roughly 100) members (editors and reviewers), we have advertised ReScience at several conferences worldwide, gave some interviews and we published an article introducing ReScience in PeerJ. Based on our experience at managing the journal during these three years, we think the time is ripe for proposing some changes.}
\def \replicationCITE{}
\def \replicationBIB{}
\def \replicationURL{}
\def \replicationDOI{}
\def \contactNAME{Mathieu Bourdenx}
\def \contactEMAIL{mathieu.bourdenx@u-bordeaux.fr}
\def \articleKEYWORDS{python, computational model, local translation, protein dynamicsx}
\def \journalNAME{ReScience C}
\def \journalVOLUME{4}
\def \journalISSUE{1}
\def \articleNUMBER{}
\def \articleDOI{}
\def \authorsFULL{Mathieu Bourdenx}
\def \authorsABBRV{M. Bourdenx}
\def \authorsSHORT{Bourdenx}
\title{\articleTITLE}
\date{}
\author[1,2,\orcid{0000-0002-9799-9222}]{Mathieu Bourdenx}
\affil[1]{Univ. de Bordeaux, Institut des Maladies Neurodégénératives, UMR 5293, Bordeaux, France}
\affil[2]{CNRS, Institut des Maladies Neurodégénératives, UMR 5293, Bordeaux, France}
